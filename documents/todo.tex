\documentclass{article}
\usepackage{amssymb}
\usepackage{ifthen}
\usepackage{fontspec}
\usepackage{hyperref}

\setmainfont[SizeFeatures={Size=7}]{JetBrainsMonoNL-Regular}
\newcommand{\todoitem}[2][notdone]{\noindent\ifthenelse{\equal{#1}{done}}{\noindent\textbf{\ensuremath{\checkmark}}}{\noindent\textbf{\ensuremath{\square}}}\hspace*{0.2em}\small#2\\}
\newcommand{\subtodoitem}[2][notdone]{\noindent\ifthenelse{\equal{#1}{done}}{\hspace*{1em}\textbf{\ensuremath{\checkmark}}}{\hspace*{1em}\textbf{\ensuremath{\square}}}\hspace*{0.2em}\small#2\\}
\newcommand{\subsubtodoitem}[2][notdone]{\noindent\ifthenelse{\equal{#1}{done}}{\hspace*{2em}\textbf{\ensuremath{\checkmark}}}{\hspace*{2em}\textbf{\ensuremath{\square}}}\hspace*{0.2em}\small#2\\}
\newcommand{\subsubsubtodoitem}[2][notdone]{\noindent\ifthenelse{\equal{#1}{done}}{\hspace*{3em}\textbf{\ensuremath{\checkmark}}}{\hspace*{3em}\textbf{\ensuremath{\square}}}\hspace*{0.2em}\small#2\\}
\newcommand{\subsubsubsubtodoitem}[2][notdone]{\noindent\ifthenelse{\equal{#1}{done}}{\hspace*{4em}\textbf{\ensuremath{\checkmark}}}{\hspace*{4em}\textbf{\ensuremath{\square}}}\hspace*{0.2em}\small#2\\}
\newcommand{\subsubsubsubsubtodoitem}[2][notdone]{\noindent\ifthenelse{\equal{#1}{done}}{\hspace*{5em}\textbf{\ensuremath{\checkmark}}}{\hspace*{5em}\textbf{\ensuremath{\square}}}\hspace*{0.2em}\small#2\\}
\newcommand{\explanation}[2]{\hspace*{\dimexpr#2em\relax} \small#1\\}

\begin{document}

\title{ChainGrid TODO list}
\author{}
\date{}

% \section{ChainGrid TODO list}
\todoitem{Chaingrid}
    \subtodoitem[done]{Global initialization}
    \subtodoitem{Rendering}
        \subsubtodoitem{Vulkan}
            \subsubsubtodoitem{TODO}
                \explanation{Not doing vulkan graphics right now, so not needed.}{4}
            \subsubsubtodoitem{Initialization}
            \subsubsubtodoitem{Begin rendering}
            \subsubsubtodoitem{End rendering}
            \subsubsubtodoitem{Quad rendering}
            \subsubsubtodoitem{Text rendering}
            \subsubsubtodoitem{Shaders}
        \subsubtodoitem{OpenGL}
            \subsubsubtodoitem[done]{Initialization}
                \subsubsubsubtodoitem[done]{Make context}
                    \explanation{Making an OpenGL context is essential, without it no OpenGL function calls can or will be executed}{5}
                \subsubsubsubtodoitem[done]{Enable flags}
                    \subsubsubsubsubtodoitem[done]{Enable culling}
                    \subsubsubsubsubtodoitem[done]{Enable blending}
                        \explanation{Culling is used for not rendering things not visible.}{6}
                        \explanation{Blending is essential for text and texture rendering.}{6}
            \subsubsubtodoitem[done]{Begin rendering}
                \subsubsubsubtodoitem[done]{Clear color}
                \subsubsubsubtodoitem[done]{Clear color buffer bit and depth buffer bit}
                    \explanation{We clear the first color to get it in the color stencil, the depth stencil isn't needed in 2D but still handy.}{5}
                    \explanation{After that we clear the color and depth buffer bits which send the colors and depths to the screen.}{5}
            \subsubsubtodoitem[done]{End rendering}
                \subsubsubsubtodoitem[done]{GLFW swap buffers}
                    \explanation{GLFW handles the screen for us which also means that it should handle the swapping of buffers for us.}{5}
            \subsubsubtodoitem{Texture rendering}
                % \subsubsubsubtodoitem{TODO}
            \subsubsubtodoitem[done]{Text rendering}
                % \subsubsubsubtodoitem{TODO}
            \subsubsubtodoitem[done]{Quad rendering}
                \subsubsubsubtodoitem[done]{Shader}
                \subsubsubsubtodoitem[done]{Normalized Coordinates}
                \subsubsubsubtodoitem[done]{VAO}
                \subsubsubsubtodoitem[done]{VBO}
            \subsubsubtodoitem[done]{Shaders}
                \subsubsubsubtodoitem[done]{Uniforms}
                    \subsubsubsubsubtodoitem[done]{Matrices}
                    \subsubsubsubsubtodoitem[done]{Vectors}
                    \subsubsubsubsubtodoitem[done]{Sampler2D}
                \subsubsubsubtodoitem[done]{Layout}
                    \subsubsubsubsubtodoitem[done]{Vectors}
            \subsubsubtodoitem[done]{Vao}
                \subsubsubsubtodoitem[done]{Generate}
                \subsubsubsubtodoitem[done]{Destroy}
                \subsubsubsubtodoitem[done]{Bind}
                    \explanation{Binding is necessary with every OpenGL Object, without it no data is sent to the GPU}{5}
                \subsubsubsubtodoitem[done]{Unbind}
                \subsubsubsubtodoitem[done]{Set attribute(s)}
            \subsubsubtodoitem[done]{Vbo}
                \subsubsubsubtodoitem[done]{Generate}
                    \explanation{Every Vertex Buffer Object (VBO) needs to be generated every drawing call.}{5}
                    \explanation{This is because every call the data passed in is different.}{5}
                \subsubsubsubtodoitem[done]{Destroy}
                    \explanation{We need to destroy every Vertex Buffer Object (VBO) because otherwise we would just be wastig GPU recources.}{5}
                \subsubsubsubtodoitem[done]{Bind}
                    \explanation{Binding is necessary with every OpenGL Object, without it no data is sent to the GPU}{5}
                \subsubsubsubtodoitem[done]{Unbind}
                    \explanation{Unbinding is necessary with every OpenGL Object, otherwise we would use data we don't expect.}{5}
                \subsubsubsubtodoitem[done]{Set buffer(s)}
                    \explanation{Without being able to set a buffer, we still don't get anything rendering to the screen or sent to the GPU.}{5}
                \subsubsubsubtodoitem[done]{Set sub buffer(s)}
                    \explanation{While sub buffer(s) aren't needed in every OpenGL application, it's still nice to have.}{5}
                    \explanation{This is mainly only needed for text rendering where we update the VBO a lot but the size doesn't change}{5}
    \\
    \subtodoitem{Entities}
        \subsubtodoitem[done]{Player}
            \subsubsubtodoitem[done]{Movement}
                \subsubsubsubtodoitem[done]{Input handling}
                \subsubsubsubtodoitem[done]{New transform}
            \subsubsubtodoitem[done]{Components}
                \subsubsubsubtodoitem[done]{Collision2D}
                \subsubsubsubtodoitem[done]{Transform}
                \subsubsubsubtodoitem[done]{Drawable}
        \subsubtodoitem[done]{Wall}
            \subsubsubtodoitem[done]{Components}
                \subsubsubsubtodoitem[done]{Collision2D}
                \subsubsubsubtodoitem[done]{Transform}
                \subsubsubsubtodoitem[done]{Drawable}
        \subsubtodoitem{Start}
            \subsubsubtodoitem{Components}
                \subsubsubsubtodoitem{Collision2D}
                \subsubsubsubtodoitem{Transform}
                \subsubsubsubtodoitem{Drawable}
        \subsubtodoitem{Finish}
            \subsubsubtodoitem{Components}
                \subsubsubsubtodoitem{Collision2D}
                \subsubsubsubtodoitem{Transform}
                \subsubsubsubtodoitem{Drawable}
    \\
    \subtodoitem{UI}
        \subsubtodoitem[done]{Text}
            \subsubsubtodoitem[done]{Rendering}
                \subsubsubsubtodoitem[done]{Text rendering}
        \subsubtodoitem{Button}
            \subsubsubtodoitem{Rendering}
                \subsubsubsubtodoitem{Texture rendering}
                \subsubsubsubtodoitem{Quad rendering}
            \subsubsubtodoitem{Interaction}
    \\
    \subtodoitem{Components}
        \subsubtodoitem[done]{Collision2D}
            \subsubsubtodoitem[done]{Check intersection with other Collision2D}
            \subsubsubtodoitem[done]{On collision}
            \subsubsubtodoitem[done]{On no collision}
        \subsubtodoitem[done]{Transform}
            \subsubsubtodoitem[done]{Pos}
            \subsubsubtodoitem[done]{Size}
            \subsubsubtodoitem[done]{Scale}
            \subsubsubtodoitem[done]{Rotation}
        \subsubtodoitem{Drawable}
            \subsubsubtodoitem{Render texture}
            \subsubsubtodoitem[done]{Render quad}

\end{document}